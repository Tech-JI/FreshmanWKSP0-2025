\section{Shell}

\begin{frame}[fragile]{Introduction to shell}
	\begin{itemize}
		\item A shell is a command-line interpreter that provides a user interface for accessing an operating system's services.
		\item It allows users to execute commands, manage files, and run programs through text-based inputs.
		\item Consider it as a direct communication channel between you and your computer's operating system.
	\end{itemize}
\end{frame}

\begin{frame}[fragile]{Types of shells}
	\begin{itemize}
		\item \textbf{PowerShell (Windows)}: A task automation and configuration management framework from Microsoft.
		\item \textbf{Bash (Linux/Mac)}: The Bourne Again Shell, the default shell on most Linux distributions and older macOS versions.
		\item \textbf{Zsh (Mac/Linux)}: An extended version of Bash with additional features like better auto-completion and theme support.
	\end{itemize}
\end{frame}

\begin{frame}[fragile]{Why zsh?}
	\begin{itemize}
		\item Enhanced auto-completion for commands, file paths, and options
		\item Better customization options with themes and plugins
		\item Improved file globbing and array handling
		\item Spelling correction and approximate completion
		\item Built-in support for Git and other version control systems
		\item macOS has made zsh the default shell since Catalina (10.15)
	\end{itemize}
\end{frame}
\begin{frame}[fragile]{WSL installation}
	\begin{itemize}
		\item For Macos and Linux users, you can take a rest
		\item For windows users, check the wsl installation manual wsl.pdf
	\end{itemize}
\end{frame}

\begin{frame}[fragile]{File organization in Linux}
	\begin{figure}
		\begin{center}
			\includegraphics[width=0.95\textwidth]{figures/linux_file_sys.png}
		\end{center}
	\end{figure}
	
\end{frame}

\begin{frame}[fragile]{File organization in Linux}
	\begin{itemize}
		\item \textbf{/}: Root directory - the base of the entire file system
		\item \textbf{/home}: User home directories (your personal files)
		\item \textbf{/etc}: System configuration files
		\item \textbf{/usr}: User programs and support files
		\item \textbf{/var}: Variable data like logs, databases, websites
		\item \textbf{/tmp}: Temporary files
		\item \textbf{/bin}: Essential command binaries
		\item \textbf{/lib}: Essential shared libraries and kernel modules
		\item \textbf{/dev}: Device files
		\item \textbf{/proc}: Process information and system information
	\end{itemize}

	For more detailed information, check \href{https://en.wikipedia.org/wiki/Filesystem_Hierarchy_Standard}{this wiki}
\end{frame}

\begin{frame}[fragile]{Basic bash commands}
	\begin{itemize}
		\item \textbf{pwd}: Print working directory - shows your current location in the file system
		\item \textbf{ls}: List directory contents (files and folders)
		\item \textbf{cd}: Change directory - navigate between folders
		\item \textbf{mkdir}: Create a new directory
		\item \textbf{touch}: Create an empty file or update file timestamps
		\item \textbf{cp}: Copy files or directories
		\item \textbf{mv}: Move or rename files or directories
		\item \textbf{rm}: Remove files or directories
		\item \textbf{cat}: Display file contents
		\item \textbf{echo}: Print text or variables to the terminal
		\item \textbf{>}: Redirect stdout to overwrite file
		\item \textbf{>>}: Redirect stdout to append file
	\end{itemize}
\end{frame}

\begin{frame}[fragile]{Practical examples}
	\begin{minted}{bash}
# Navigate to your home directory
cd ~

# List files in long format
ls -l

# Create a new directory
mkdir my_project

# Navigate into the directory
cd my_project

# Create a new file
touch README.md

# Copy a file
cp README.md README_copy.md

# Move/Rename a file
mv README_copy.md README_backup.md

# Remove a file
rm README_backup.md
	\end{minted}
\end{frame}

\begin{frame}[fragile]{Practical examples}
	\begin{minted}{bash}
cd ~/my_project
# redirect stdout into files
echo "fooo" >foo
echo "barrr" >bar

# concatenate two files
cat foo bar

# Go to parent path
cd ..

# Dangerous!!! You'd better use project like trash-cli
rm -rf my_project
	\end{minted}
\end{frame}

\begin{frame}[fragile]{Environment Variable}
	\begin{itemize}
		\item We know that, ls is at /bin/ls, but we can directly use ls command.
		\item This is the benefit of PATH environment variable.
		\item The PATH define where OS finds the executable files.
		\item You can use \textbf{echo \$PATH} to check your current PATH
	\end{itemize}
\end{frame}

\begin{frame}[fragile]{Environment Variables}
    \begin{itemize}
        \item What are environment variables?
        \item Environment variables, often called \textbf{ENVs}, are dynamic values that play a crucial role in controlling the behavior of programs and processes in Linux and other operating systems.
    \end{itemize}
\end{frame}

\begin{frame}[fragile]{Environment Variables (Examples)}
	\begin{minted}{bash}
FOO="fooo"
echo $FOO
set # display all the ENVs(global as well as local)
env # display all the global ENVs

# local variable won't show, don't use " here
bash -c 'echo $FOO'

export FOO="fooo" # define a global variable

# global variable will show
bash -c 'echo $FOO'

unset FOO
	\end{minted}
\end{frame}

\begin{frame}[fragile]{Environment Variables (Examples)}
	How to set up proxy in shell
	\begin{minted}{bash}
export http_proxy="(http)|(socks5)://127.0.0.1:<port>"
export https_proxy="(http)|(socks5)://127.0.0.1:<port>"
export all_proxy="(http)|(socks5)://127.0.0.1:<port>"

curl -i www.google.com # test with google
	\end{minted}
\end{frame}
